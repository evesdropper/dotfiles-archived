% !TeX encoding = UTF-8
% Ce fichier contient le code de l'extension "xstring"
%
%%%%%%%%%%%%%%%%%%%%%%%%%%%%%%%%%%%%%%%%%%%%%%%%%%%%%%%%%%%%%%%%%%%%%%
%                                                                    %
\def\xstringname               {xstring}                             %
\def\xstringversion              {1.84}                              %
%                                                                    %
\def\xstringdate              {2021/07/21}                           %
%                                                                    %
%%%%%%%%%%%%%%%%%%%%%%%%%%%%%%%%%%%%%%%%%%%%%%%%%%%%%%%%%%%%%%%%%%%%%%
%                                                                    %
% Author     : Christian Tellechea                                   %
% Status     : Maintained                                            %
% Maintainer : Christian Tellechea                                   %
% Email      : unbonpetit@netc.fr                                    %
% Package URL: https://www.ctan.org/pkg/xstring                      %
% Bug tracker: https://framagit.org/unbonpetit/xstring/issues        %
% Repository : https://framagit.org/unbonpetit/xstring/tree/master   %
% Copyright  : Christian Tellechea 2008-2019                         %
% Licence    : Released under the LaTeX Project Public License v1.3c %
%              or later, see http://www.latex-project.org/lppl.txt   %
% Files      : 1) xstring.tex                                        %
%              2) xstring.sty                                        %
%              3) xstring-fr.tex                                     %
%              4) xstring-fr.pdf                                     %
%              5) xstring-en.tex                                     %
%              6) xstring-en.pdf                                     %
%              7) README                                             %
%%%%%%%%%%%%%%%%%%%%%%%%%%%%%%%%%%%%%%%%%%%%%%%%%%%%%%%%%%%%%%%%%%%%%%
\ifdefined\usepackage\else
	\immediate\write-1 {Package: xstring \xstringdate\space v\xstringversion\space String manipulations (CT)}%
\fi
\newcount\integerpart
\newcount\decimalpart% compteurs utilisés par xstring

\edef\restorexstringcatcode{\catcode\number`\_ =\number\catcode`\_\relax}
\catcode`\_=11

\long\def\xs_execfirst#1#2{#1}
\long\def\xs_execsecond#1#2{#2}
\long\def\xs_gobarg#1{}
\expandafter\def\expandafter\xs_gobspace\space{}
\def\xs_deftok#1#2{\let#1= #2\relax}\xs_deftok\xs_sptoken{ }
\long\def\xs_ifnxttok#1#2#3{\xs_deftok\xs_toksmatch{#1}\def\__truecode{#2}\def\__falsecode{#3}\xs_ifnxttok_a}
\def\xs_ifnxttok_a{\futurelet\__futurtok\xs_ifnxttok_b}
\def\xs_ifnxttok_b{%
	\xs_ifx{\__futurtok\xs_sptoken}
		{\afterassignment\xs_ifnxttok_a\let\__futurtok= }
		{\xs_ifx{\__futurtok\xs_toksmatch}\__truecode\__falsecode}%
}
\def\xs_ifstar#1{\xs_ifnxttok*{\xs_execfirst{#1}}}
\long\def\xs_testopt#1#2{\xs_ifnxttok[{#1}{#1[{#2}]}}
\long\def\xs_exparg#1#2{\expandafter\xs_exparg_a\expandafter{#2}{#1}}%
\long\def\xs_exparg_a#1#2{#2{#1}}
\long\def\xs_exptwoargs#1#2#3{\xs_exparg{\xs_exparg{#1}{#2}}{#3}}
\long\def\xs_eearg#1#2{\xs_exparg{\xs_exparg{#1}}{#2}}
\long\def\xs_edefaddtomacro#1#2{\edef#1{\unexpanded\expandafter{#1}#2}}
\long\def\xs_addtomacro#1#2{\xs_exparg{\def#1}{#1#2}}
\long\def\xs_eaddtomacro#1#2{\xs_exparg{\xs_addtomacro#1}{#2}}
\long\def\xs_eeaddtomacro#1#2{\xs_exparg{\xs_eaddtomacro#1}{#2}}
\long\def\xs_endgroup_eaddtomacro#1#2{\xs_exparg{\endgroup\xs_addtomacro#1}{#2}}
\long\def\xs_beforespace#1 #2\_nil{#1}
\long\def\xs_afterspace#1 #2\_nil{#2}
\long\def\xs_gobtonil#1\_nil{}
\long\def\xs_gobtobrace#1#{}
\long\def\xs_ifx#1{\ifx#1\expandafter\xs_execfirst\else\expandafter\xs_execsecond\fi}
\def\xs_ifnum#1{\ifnum#1\expandafter\xs_execfirst\else\expandafter\xs_execsecond\fi}
\long\def\xs_ifempty#1{\expandafter\xs_ifx\expandafter{\expandafter\relax\detokenize{#1}\relax}}

\def\xs_toks_toverb{% lit 1 argument et le transforme en verb
	\begingroup
		\def\do##1{\catcode`##112 }%
		\dospecials
		\obeyspaces
		\xs_read_verb
}

\def\setverbdelim#1{% définit quel est le délimiteur de verb
	\xs_exparg\xs_ifempty{\xs_gobarg#1}%
		{}
		{\errmessage{Package xstring Warning: verb delimiter "\detokenize{#1}" is not a single token}}%
	\def\xs_read_verb##1#1##2#1{\endgroup\xs_assign_verb{##2}}% on appelle l'exécution de fin
}

% Assigne l'argument entre délimiteur verb dans la sc #1
\def\verbtocs#1{%
	\def\xs_assign_verb##1{\def#1{##1}}%
	\xs_toks_toverb
}

\begingroup
	\catcode0 3 \gdef\xs_twozeroschars{^^00}%
	\catcode0 7 \xdef\xs_twozeroschars{\xs_twozeroschars^^00}%
\endgroup

\edef\xs_reserved_A{\long\def\noexpand\xs_assigntokenizeresult##1\xs_twozeroschars}
\xs_reserved_A#2{\endgroup\xs_exparg{\def#2}{\xs_gobarg#1}}

\def\tokenize#1#2{%
	\begingroup
		\xs_def\xs_reserved_A{#2}% on développe en accord avec \fullexpandarg ou \noexpandarg
		\everyeof\expandafter{\xs_twozeroschars#1}%
		\endlinechar-1
		\expandafter\xs_assigntokenizeresult\scantokens\expandafter{\expandafter\relax\xs_reserved_A}% on fait l'assignation
}%

\long\def\xs_assign_xstringresult#1#2{\xs_ifempty{#2}{#1}{\def#2{#1}}}

\def\xs_ldef{\long\def}
\def\xs_ledef{\long\edef}

\def\normalexpandarg{%
	\let\xs_def\xs_ldef% on définit \xs_call avec \def
	\def\xs_expand##1{\unexpanded\expandafter{##1}}%
}
\let\noexpandarg\normalexpandarg% synonyme

\def\expandarg{%
	\let\xs_def\xs_ldef% on définit \xs_call avec \def
	\def\xs_expand##1{\unexpanded\expandafter\expandafter\expandafter{##1}}%
}

\def\fullexpandarg{%
	\let\xs_def\xs_ledef% on définit\xs_call avec \edef
	\def\xs_expand##1{##1}% et on neutralise \xs_expand
}

\def\saveexpandmode{\let\xs_saved_def\xs_defarg\let\xs_saved_expand\xs_expand}
\def\restoreexpandmode{\let\xs_defarg\xs_saved_def\let\xs_expand\xs_saved_expand}

% Macro interne renvoyant #2 si son argument commence par une accolade ouvrante "{"
% et #3 sinon (Ulrich Diez sur comp.text.tex)
\long\def\xs_ifbeginwithbrace#1{%
	\csname xs_exec%
		\expandafter\xs_gobarg\string{%
		\expandafter\xs_gobarg\expandafter{\expandafter{\string#1}%
		\expandafter\expandafter\expandafter\expandafter\expandafter\expandafter\expandafter\expandafter\expandafter\expandafter\expandafter\expandafter\expandafter\expandafter\expandafter\xs_execfirst
		\expandafter\expandafter\expandafter\expandafter\expandafter\expandafter\expandafter\xs_gobarg
		\expandafter\expandafter\expandafter\xs_gobarg
		\expandafter\expandafter\expandafter{\expandafter\string\expandafter}\string}%
		\expandafter\xs_gobarg\string}%
		\xs_execsecond{first}{second}%
	\endcsname
}

% Cette macro interne prend la 1ere unité syntaxique de #1 et assigne le résultat à #2
\long\def\xs_returnfirstsyntaxunit#1#2{%
	\let\xs_toks\relax
	\xs_ifempty{#1}
		{\let#2\empty
		}
		{\xs_eearg\xs_ifempty{\expandafter\xs_beforespace\detokenize{#1} \_nil}% #1 commence par un espace ?
			{\def#2{ }}
			{\xs_ifbeginwithbrace{#1}
				{\def\xs_groupfound{\xs_exparg{\def#2}{\expandafter{#2}}\xs_gobtonil}% on met #2 dans des accolades et on finit
				\let\xs_toks\bgroup
				\afterassignment\xs_groupfound
				\def#2%
				}
				{\long\def\xs_assignfirstarg##1##2\_nil{\def#2{##1}}%
				\xs_assignfirstarg
				}%
			#1\_nil
			}%
		}%
}

\long\def\xs_removefirstsyntaxunit#1#2{%
	\xs_ifempty{#1}
		{\let#2\empty
		}
		{\xs_eearg\xs_ifempty{\expandafter\xs_beforespace\detokenize{#1} \_nil}% #1 commence par un espace ?
			{\xs_exparg\xs_ifbeginwithbrace{\xs_gobspace#1}
				{\xs_exparg{\def#2}{\xs_gobtobrace#1}}%
				{\xs_exparg{\def#2}{\xs_afterspace#1\_nil}}%
			}%
			{\xs_exparg{\def#2}{\xs_gobarg#1}}%
		}%
}

% Cette macro interne est utilisée dans les macros étoilées pour :
% 1) développer l'argument selon qu'on a choisit \fullexpandarg
%    ou \normalexpandarg, et ceci à l'aide de la macro \xs_def
% 2) Ensuite, on détokenize ce développement de façon n'avoir plus que
%    des catcodes de 10 pour les espaces et 12 pour le reste.
\long\def\xs_expand_and_detokenize#1#2{%
	\long\def#1{#2}%
	\expandafter\edef\expandafter#1\expandafter{\xs_expand#1}% on développe #2 selon le mode de développement
	\long\edef#1{\detokenize\expandafter{#1}}% puis on détokenize et on assigne à #1
}

\long\def\xs_expand_and_assign#1#2{\xs_def#1{#2}}% on développe #2 selon \fullexpandarg ou \normalexpandarg

\def\xs_arglist{0########1########2########3########4########5########6########7########8########9}

% \xs_assign_args{3} met dans la sc \xs_myarg les tokens "####1####2####3"
\def\xs_assign_args#1{\def\xs_assign_args_i0##1#1##2\_nil{\def\xs_myarg{##1#1}}\expandafter\xs_assign_args_i\xs_arglist\_nil}

% \xs_assign_opt_args{3} met dans la sc \xs_myarg les tokens "[####1]####2####3####4"
\def\xs_assign_opt_args#1{\xs_exparg\xs_assign_opt_args_i{\number\numexpr#1+1}}
\def\xs_assign_opt_args_i#1{\def\xs_assign_opt_args_ii0##11##2#1##3\_nil{\def\xs_myarg{[##11]##2#1}}\expandafter\xs_assign_opt_args_ii\xs_arglist\_nil}

% \xs_assign_one_arg{3} met dans la sc \xs_myarg les tokens "####3"
\def\xs_assign_one_arg#1{\xs_exparg\xs_assign_one_arg_i{\number\numexpr#1-1}{#1}}
\def\xs_assign_one_arg_i#1#2{\def\xs_assign_one_arg_ii##1#1##2#2##3\_nil{\def\xs_myarg{##2#2}}\expandafter\xs_assign_one_arg_ii\xs_arglist\_nil}

% #1 : numéro du 1er argument; #2 : nombre de lignes; #3 argument optionnel; #4 : dernier numéro de l'argument devant être détokenisé si étoile
\def\xs_buildlines#1#2#3#4{%
	\let\xs_newlines\empty
	\let\xs_newargs\empty
	\def\xs_buildlines_i##1{%
		\xs_exparg\xs_assign_one_arg{\number\numexpr##1+#1-1}%
		\edef\xs_reserved_B{\noexpand\xs_expand\csname xs_arg_\romannumeral\numexpr##1\endcsname}%
		\xs_ifnum{##1=1 }% si c'est le premier argument
			{\xs_ifempty{#3}%
				{\xs_eaddtomacro\xs_newargs{\expandafter{\xs_reserved_B}}%
				\edef\xs_reserved_B{\ifnum##1>#4 xs_def\else xs_assign\fi}%
				}% et s'il y a un argument optionnel alors, on met des crochets
				{\xs_eaddtomacro\xs_newargs{\expandafter[\xs_reserved_B]}%
				\def\xs_reserved_B{xs_def}% ne pas détokéniser l'argument optionnel grace au \xs_def
				}%
			}
			{\xs_eaddtomacro\xs_newargs{\expandafter{\xs_reserved_B}}%
			\edef\xs_reserved_B{\ifnum##1>#4 xs_def\else xs_assign\fi}%
			}%
		\edef\xs_newlines{\unexpanded\expandafter{\xs_newlines}\expandafter\noexpand\csname\xs_reserved_B\endcsname\expandafter\noexpand\csname xs_arg_\romannumeral\numexpr##1\endcsname{\xs_myarg}}%
		\xs_ifnum{##1<#2 }
			{\xs_exparg\xs_buildlines_i{\number\numexpr##1+1}}
			{}%
	}%
	\xs_buildlines_i1%
}

\def\xs_newmacro{%
	\xs_ifstar
		{\let\xs_reserved_D\empty\xs_newmacro_}
		{\let\xs_reserved_D\relax\xs_newmacro_0}%
}

% #1 : dernier numéro de l'argument devant être détokenisé
% #2 : nom de la macro publique
% #3 : argument optionnel (vide si pas d'arg optionnel)
% #4 : nombre d'arguments obligatoires
% #5 : utilisation de testopt (1 si oui, 0 si non)
% #6 : code de la macro
\def\xs_newmacro_#1#2#3#4#5{%
	\edef\xs_reserved_A{xs_\expandafter\xs_gobarg\string#2}%
	\edef\xs_reserved_C{\expandafter\noexpand\csname\xs_reserved_A _\ifx\empty#3\empty _\fi\endcsname}%
	\edef\xs_reserved_B{%
		\long
		\ifx\empty\xs_reserved_D
			\def\noexpand#2{\noexpand\xs_ifstar
				{\let\noexpand\xs_assign\noexpand\xs_expand_and_detokenize\expandafter\noexpand\xs_reserved_C}%
				{\let\noexpand\xs_assign\noexpand\xs_expand_and_assign\expandafter\noexpand\xs_reserved_C}%
			}%
		\else
			\def\noexpand#2{\let\noexpand\xs_assign\noexpand\xs_expand_and_assign\expandafter\noexpand\xs_reserved_C}%
		\fi
		\ifx\empty#3\empty
		\else
			\def\expandafter\noexpand\xs_reserved_C{%
				\noexpand\xs_testopt{\expandafter\noexpand\csname\xs_reserved_A __\endcsname}{\ifx\xs_def\xs_ledef#3\else\unexpanded{#3}\fi}}%
		\fi
	}%
	% Que fait \xs_reserved_B ? Il définit :
	% si #3 est vide : \NOM{\xs_ifstar{\let\xs_assign\xs_expand_and_detokenize\xs_NOM__}{\let\xs_assign\xs_expand_and_assign\xs_NOM__}}
	% si #3 existe   : \NOM{\xs_ifstar{\let\xs_assign\xs_expand_and_detokenize\xs_NOM_}{\let\xs_assign\xs_expand_and_assign\xs_NOM_}}
	%                  \xs_NOM_{\xs_testopt{\xs_NOM__}{#3}}}
	\xs_reserved_B
	\ifx\empty#3\empty
		\xs_buildlines1{#4}{#3}{#1}%
		\xs_assign_args{#4}%
	\else
		\xs_exparg{\xs_buildlines1}{\number\numexpr#4+1}{#3}{#1}%
		\xs_assign_opt_args{#4}%
	\fi
	\edef\xs_reserved_B{\def\expandafter\noexpand\csname\xs_reserved_A __\endcsname\xs_myarg}%
	\edef\xs_reserved_C{\unexpanded\expandafter{\xs_newlines}\edef\noexpand\xs_call}%
	\edef\xs_reserved_D{\noexpand\noexpand\expandafter\noexpand\csname\xs_reserved_A\endcsname\unexpanded\expandafter{\xs_newargs}}%
	\ifnum#5=1 \edef\xs_reserved_D{\noexpand\noexpand\noexpand\xs_testopt{\unexpanded\expandafter{\xs_reserved_D}}{}}\fi
	\xs_edefaddtomacro\xs_reserved_C{{\unexpanded\expandafter{\xs_reserved_D}}\noexpand\xs_call}%
	\xs_edefaddtomacro\xs_reserved_B{{\unexpanded\expandafter{\xs_reserved_C}}}%
	% Que fait \xs_reserved_B ? Il définit par exemple pour 3 arguments obligatoires et 1 facultatif :
	%    \def\xs_NOM__[##2]##3##4##5{%
	%          \xs_def\xs_arg_i{##2}\xs_assign\xs_arg_ii{##3}\xs_assign\xs_arg_iii{##4}\xs_asign\xs_arg_iv{##5}%
	% si #5=0: \edef\xs_call{\noexpand\xs_NOM[\xs_expand\xs_arg_i]{\xs_expand\xs_arg_ii}{\xs_expand\xs_arg_iii}{\xs_expand\xs_arg_iv}}%
	% si #5=1: \edef\xs_call{\noexpand\xs_testopt{\noexpand\xs_NOM[\xs_expand\xs_arg_i]{\xs_expand\xs_arg_i}{\xs_expand\xs_arg_ii}{\xs_expand\xs_arg_iii}{\xs_expand\xs_arg_iv}}{}}%
	%          \xs_call}
	\long\xs_reserved_B
	\edef\xs_reserved_B{%
		\def\expandafter\noexpand\csname\xs_reserved_A\endcsname
			\xs_myarg\ifnum#5=1 [\unexpanded{##}\number\numexpr\ifx\empty#3\empty#4+1\else#4+2\fi]\fi
	}%
	% Que fait \xs_reserved_B ? Il définit par exemple pour 3 arguments obligatoires et 1 facultatif :
	%    \def \xs_NOM[##2]##3##4##5[##6]{#6}
	\long\xs_reserved_B
}

% macro générique qui lit \xs_reserved_C us par us
% 3 sous-routines sont appelées à des moments clé :
% \xs_atendofgroup      (un groupe se finit, appel récursif)
% \xs_atbegingroup      (un groupe vient d'être ouvert)
% \xs_atnextsyntaxunit  (la future US n'est pas un groupe)
\def\xs_read_reserved_C{%
	\xs_exparg\xs_ifempty{\xs_reserved_C}%
		{\xs_ifnum{\xs_nestlevel=0 }
			{}
			{\xs_atendofgroup
			\xs_read_reserved_C
			}%
		}
		{\xs_exparg\xs_returnfirstsyntaxunit{\xs_reserved_C}\xs_reserved_A
		\xs_exparg\xs_removefirstsyntaxunit{\xs_reserved_C}\xs_reserved_C
		\xs_exploregroups
		\xs_ifx{\bgroup\xs_toks}
			{\advance\integerpart1
			\begingroup
				\expandafter\def\expandafter\xs_reserved_C\xs_reserved_A
				\xs_calc_grouIP
				\def\xs_nestlevel{1}%
				\integerpart0
				\xs_atbegingroup
			}
			{\advance\decimalpart1
			\xs_atnextsyntaxunit
			}%
		\xs_read_reserved_C
		}%
}

% macro générique qui lit \xs_reserved_D en procédant à des tests avec \IfBeginWith
% 1 sous-routine est appelée :
% \xs_atoccurfound      (une occurrence a été trouvée)
\def\xs_read_reserved_D{%
	\xs_exparg\xs_ifempty{\xs_reserved_D}%
		{\xs_ifnum{\xs_nestlevel=0 }
			{}
			{\edef\__temp{%
				\endgroup
				\noexpand\decimalpart=\number\decimalpart\relax
				\unexpanded{\xs_addtomacro\xs_reserved_C}{{\unexpanded\expandafter{\xs_reserved_C}}}%
				}%
			\__temp
			\xs_read_reserved_D
			}%
		}%
		{\xs_exptwoargs\xs_IfBeginWith_i\xs_reserved_D\xs_reserved_E
			{\advance\decimalpart1
			\let\xs_reserved_D\xs_reserved_A
			\xs_atoccurfound
			}%
			{\xs_exparg\xs_returnfirstsyntaxunit{\xs_reserved_D}\xs_reserved_A
			\xs_exparg\xs_removefirstsyntaxunit{\xs_reserved_D}\xs_reserved_D
			\xs_exploregroups
			\xs_ifx{\bgroup\xs_toks}
				{\advance\integerpart1
				\begingroup
					\expandafter\def\expandafter\xs_reserved_D\xs_reserved_A
					\xs_calc_grouIP
					\let\xs_reserved_C\empty
					\def\xs_nestlevel{1}%
					\integerpart0
				}
				{\xs_eaddtomacro\xs_reserved_C{\xs_reserved_A}%
				}%
			}%
		\xs_read_reserved_D
		}%
}

\xs_newmacro\StrRemoveBraces{}{1}{1}{%
	\def\xs_reserved_C{#1}%
	\let\xs_reserved_B\empty
	\def\xs_nestlevel{0}%
	\xs_StrRemoveBraces_i
	\xs_exparg\xs_assign_xstringresult{\xs_reserved_B}{#2}%
}

\def\xs_StrRemoveBraces_i{%
	\xs_exparg\xs_ifempty{\xs_reserved_C}%
		{\xs_ifnum{\xs_nestlevel=0 }
			{}
			{\xs_endgroup_eaddtomacro\xs_reserved_B\xs_reserved_B
			\xs_StrRemoveBraces_i
			}%
		}
		{\xs_exparg\xs_returnfirstsyntaxunit{\xs_reserved_C}\xs_reserved_A
		\xs_exparg\xs_removefirstsyntaxunit{\xs_reserved_C}\xs_reserved_C
		\xs_ifx{\bgroup\xs_toks}
			{\xs_ifx{\xs_exploregroups\relax}% on explore les groupes ?
				{\begingroup
					\expandafter\def\expandafter\xs_reserved_C\xs_reserved_A
					\def\xs_nestlevel{1}%
					\let\xs_reserved_B\empty
				}
				{\expandafter\xs_addtomacro\expandafter\xs_reserved_B\xs_reserved_A
				}%
			}
			{\xs_eaddtomacro\xs_reserved_B{\xs_reserved_A}%
			}%
		\xs_StrRemoveBraces_i
		}%
}

% \xs_cutafteroccur coupe l'argument #1 après la #3ème occurrence de #2
% \xs_reserved_C : contient ce qui se trouve avant cette occurrence
% \xs_reserved_D : contient ce qui se trouve avant cette occurrence y compris cette occurrence
% \xs_reserved_E : contient ce qui se trouve après l'occurrence
% si l'occurrence n'existe pas ou qu'un des arguments est vide, toutes les chaines renvoyées sont vides
\long\def\xs_cutafteroccur#1#2#3{%
	\xs_ifnum{#3<1 }
		{\let\xs_reserved_C\empty\let\xs_reserved_E\empty\gdef\groupID{}}
		{\xs_cutafteroccur_i{#1}{#2}{#3}}%
}
	
\long\def\xs_cutafteroccur_i#1#2#3{%
	\def\xs_reserved_D{#1}\let\xs_reserved_C\empty\def\xs_reserved_E{#2}%
	\decimalpart0 \integerpart0 \gdef\groupID{0}\def\xs_nestlevel{0}%
	\def\xs_atoccurfound{%
		\xs_ifnum{\decimalpart=\numexpr(#3)\relax}
			{\edef\__temp{%
				\def\noexpand\xs_reserved_C{\unexpanded\expandafter{\xs_reserved_C}}%
				\def\noexpand\xs_reserved_D{\unexpanded\expandafter{\xs_reserved_D}}%
				\noexpand\decimalpart=\number\decimalpart\relax
				}%
			\expandafter\xs_exitallgroups\__temp
			\xs_gobarg% mange l'appel récursif \xs_read_reserved_D dans \xs_read_reserved_D
			}
			{\xs_eaddtomacro\xs_reserved_C{\xs_reserved_E}%
			}%
	}%
	\xs_read_reserved_D
	\xs_ifnum{\decimalpart=\numexpr(#3)\relax}% occurrence trouvée ?
		{\let\xs_reserved_E\xs_reserved_D
		\xs_exparg{\def\xs_reserved_D}{\xs_reserved_C#2}%
		}
		{\let\xs_reserved_C\empty
		\let\xs_reserved_E\empty
		\gdef\groupID{}%
		}%
}

\xs_newmacro*3\IfSubStr{1}{2}{0}{%
	\xs_cutafteroccur{#2}{#3}{#1}%
	\xs_exparg\xs_ifempty{\xs_reserved_D}\xs_execsecond\xs_execfirst
}

\xs_newmacro*2\IfBeginWith{}{2}{0}{%
	\xs_ifempty{#2}%
		{\xs_execsecond}
		{\xs_IfBeginWith_i{#1}{#2}}%
}

\long\def\xs_IfBeginWith_i#1#2{%
	\xs_ifempty{#2}%
		{\xs_execfirst% #2 est vide, tous les tests sont passés avec succès : on renvoie #3
		}
		{\xs_ifempty{#1}%
			{\xs_execsecond% #1 est vide, c'est que #2 est + long que #1 : on renvoie #4
			}
			{\xs_returnfirstsyntaxunit{#2}\xs_reserved_B
			\xs_returnfirstsyntaxunit{#1}\xs_reserved_A
			\xs_ifx{\xs_reserved_A\xs_reserved_B}% il y a égalité...
				{\xs_removefirstsyntaxunit{#2}\xs_reserved_B
				\xs_removefirstsyntaxunit{#1}\xs_reserved_A% on enlève les 1ere unités syntaxiques
				\xs_exptwoargs\xs_IfBeginWith_i\xs_reserved_A\xs_reserved_B
				}
				{\xs_execsecond
				}%
			}%
		}%
}

\xs_newmacro*2\IfEndWith{}{2}{0}{%
	\xs_ifempty{#2}%
		{\xs_execsecond
		}
		{\xs_StrCount{#1}{#2}[\xs_reserved_A]%
		\xs_ifnum{\xs_reserved_A=0 }
			{\xs_execsecond}
			{\xs_exparg\xs_ifempty{\xs_reserved_C}}%
		}%
}

\xs_newmacro*4\IfSubStrBefore{1,1}{3}{0}{\xs_IfSubStrBefore_i[#1]{#2}{#3}{#4}}

\def\xs_IfSubStrBefore_i[#1,#2]#3#4#5{%
	\def\xs_reserved_C{#3}%
	\xs_ifx{\xs_exploregroups\relax}% si on explore les groupes
		{\let\xs_reserved_B\empty
		\def\xs_nestlevel{0}%
		\xs_StrRemoveBraces_i% on retire les accolades
		\let\xs_reserved_C\xs_reserved_B
		}
		{}%
	\xs_exparg\xs_cutafteroccur\xs_reserved_C{#5}{#2}%
	\xs_exparg\xs_cutafteroccur\xs_reserved_C{#4}{#1}%
	\gdef\groupID{}%
	\xs_exparg\xs_ifempty{\xs_reserved_C}%
		\xs_execsecond
		\xs_execfirst
}

\xs_newmacro*4\IfSubStrBehind{1,1}{3}{0}{\xs_IfSubStrBehind_i[#1]{#2}{#3}{#4}}
\long\def\xs_IfSubStrBehind_i[#1,#2]#3#4#5{\xs_IfSubStrBefore_i[#2,#1]{#3}{#5}{#4}}

\long\def\xs_formatnumber#1#2{%
	\xs_ifempty{#1}%
		{\def#2{0X}% si vide, renvoie 0X
		}
		{\xs_returnfirstsyntaxunit{#1}\xs_reserved_A
		\def\xs_reserved_B{+}%
		\xs_ifx{\xs_reserved_A\xs_reserved_B}
			{\xs_removefirstsyntaxunit{#1}\xs_reserved_C
			\xs_exparg\xs_ifempty{\xs_reserved_C}%
				{\def#2{+0X}}
				{\xs_exparg{\def#2}{\expandafter+\expandafter0\xs_reserved_C}}%
			}
			{\def\xs_reserved_B{-}%
			\xs_ifx{\xs_reserved_A\xs_reserved_B}
				{\xs_removefirstsyntaxunit{#1}\xs_reserved_A
				\xs_exparg\xs_ifempty{\xs_reserved_A}%
					{\def#2{-0X}}%
					{\xs_exparg{\def#2}{\expandafter-\expandafter0\xs_reserved_A}}%
				}
				{\def#2{0#1}%
				}%
			}%
		}%
}

\xs_newmacro\IfInteger{}{1}{0}{%
	\xs_formatnumber{#1}\xs_reserved_A
	\decimalpart0
	\afterassignment\xs_defafterinteger\integerpart\xs_reserved_A\relax\_nil
	\let\xs_after_intpart\afterinteger
	\expandafter\xs_testdot\afterinteger\_nil
	\xs_ifx{\empty\afterdecimal}
		{\xs_ifnum{\decimalpart=0 }
			{\xs_execfirst% partie décimale constituée de 0 --> seul cas où on renvoie vrai
			}
			{\expandafter\xs_defafterinteger\xs_after_intpart\relax\_nil
			\xs_execsecond
			}%
		}
		{\expandafter\xs_defafterinteger\xs_after_intpart\relax\_nil
		\xs_execsecond
		}%
}

\xs_newmacro\IfDecimal{}{1}{0}{%
	\xs_formatnumber{#1}\xs_reserved_A
	\decimalpart0
	\afterassignment\xs_defafterinteger\integerpart\xs_reserved_A\relax\_nil
	\expandafter\xs_testdot\afterinteger\_nil
	\xs_ifx{\empty\afterdecimal}
}

\begingroup\catcode`\@ 11
\long\gdef\xs_defafterinteger#1\relax\_nil{\def\@xs@afterinteger{#1}\let\afterinteger\@xs@afterinteger}
\long\gdef\xs_defafterdecimal#1\_nil{\def\@xs@afterdecimal{#1}\let\afterdecimal\@xs@afterinteger}
\endgroup

\def\xs_testdot{%
	\let\xs_decsep\empty
	\xs_ifnxttok.
		{\def\xs_decsep{.}\xs_readdecimalpart}%
		{\xs_testcomma}%
}

\def\xs_testcomma{%
	\xs_ifnxttok,
		{\def\xs_dessep{,}\xs_readdecimalpart}%
		{\xs_defafterdecimal}%
}

\long\def\xs_readdecimalpart#1#2\_nil{%
	\xs_ifempty{#2}
		{\def\xs_reserved_A{0X}}
		{\def\xs_reserved_A{0#2}}%
	\afterassignment\xs_defafterinteger\decimalpart\xs_reserved_A\relax\_nil
	\expandafter\xs_defafterdecimal\afterinteger\_nil
}

% test d'égalité sur des chaines (et des chaines contenant des nombres)
\long\def\xs_IfStrEqFalse_i#1#2{\let\xs_reserved_A\xs_execsecond}
\long\def\xs_IfStrEqFalse_ii#1#2{% renvoie vrai si les 2 arg sont décimaux et s'ils sont égaux, faux sinon
	\xs_IfDecimal{#1}%
		{\xs_IfDecimal{#2}%
			{\ifdim#1pt=#2pt
				\let\xs_reserved_A\xs_execfirst
			\else
				\let\xs_reserved_A\xs_execsecond
			\fi% les 2 sont décimaux : on fait le test
			}%
			{\let\xs_reserved_A\xs_execsecond}% un seul est décimal
		}%
		{\let\xs_reserved_A\xs_execsecond}% #1 n'est pas décimal
}

\long\def\xs_TestEqual#1#2{% teste si les 2 arguments sont égaux
	\def\xs_reserved_A{#1}\def\xs_reserved_B{#2}%
	\xs_ifx{\xs_reserved_A\xs_reserved_B}
		{\let\xs_reserved_A\xs_execfirst}% égalité parfaite des 2 chaines
		{\xs_reserved_D{#1}{#2}}%
	\xs_reserved_A
}

\xs_newmacro*2\IfStrEq{}{2}{0}{% teste si les deux chaines sont égales
	\let\xs_reserved_D\xs_IfStrEqFalse_i
	\xs_TestEqual{#1}{#2}%
}

\xs_newmacro*2\IfEq{}{2}{0}{% teste si les 2 arguments (chaine ou nombre) sont égaux
	\let\xs_reserved_D\xs_IfStrEqFalse_ii
	\xs_TestEqual{#1}{#2}%
}

\def\IfStrEqCase{%
	\xs_ifstar
		{\xs_IfStringCase{\IfStrEq*}}%
		{\xs_IfStringCase\IfStrEq}%
}

\def\IfEqCase{%
	\xs_ifstar
		{\xs_IfStringCase{\IfEq*}}%
		{\xs_IfStringCase\IfEq}%
}

\long\def\xs_IfStringCase#1#2#3{% #1=macro avec ou sans étoile  #2=argument à tester  #3=liste des {<cas>}{code>}
	\long\def\xs_testcase##1##2##3\_nil{% lit les 2 premieres unités syntaxiques dans ##1 et ##2. Les autres dans ##3
		#1{#2}{##1}% compare
			{\xs_testopt{\xs_IfStringCase_ii{##2}}{}}% le test est positif, on exécute le code correspondant et on mange l'argument optionnel ## BUGfix dans 1.82
			{\xs_ifempty{##3}%
				{\xs_testopt{\xs_IfStringCase_i}{}}% s'il n'y a plus de cas, on exécute le code <autres cas> ## BUGfix dans 1.82
				{\xs_testcase##3\_nil}% sinon, on recommence avec ce qui reste
			}%
	}%
	\xs_testcase#3\_nil
}

\long\def\xs_IfStringCase_i[#1]{#1}
\long\def\xs_IfStringCase_ii#1[#2]{#1}

% Renvoie ce qui est à gauche de l'occurence n°#1 de la sous chaine #3 dans la chaine #2
\xs_newmacro*3\StrBefore{1}{2}{1}{%
	\xs_cutafteroccur{#2}{#3}{#1}%
	\xs_exparg\xs_assign_xstringresult{\xs_reserved_C}{#4}%
}

% Renvoie ce qui est à droite de l'occurence n°#1 de la sous chaine #3 dans la chaine #2
\xs_newmacro*3\StrBehind{1}{2}{1}{%
	\xs_cutafteroccur{#2}{#3}{#1}%
	\xs_exparg\xs_assign_xstringresult{\xs_reserved_E}{#4}%
}

% Renvoie ce qui est strictement compris entre les occurrences n°#1 et n°#2
% des chaines #4 et #5 dans la chaine #3
\xs_newmacro*4\StrBetween{1,1}{3}{1}{\xs_StrBetween_i[#1]{#2}{#3}{#4}[#5]}
\long\def\xs_StrBetween_i[#1,#2]#3#4#5[#6]{%
	\begingroup
		\noexploregroups
		\xs_cutafteroccur{#3}{#5}{#2}%
		\xs_exparg\xs_cutafteroccur{\xs_reserved_C}{#4}{#1}%
		\expandafter
	\endgroup
	\expandafter\xs_assign_xstringresult\expandafter{\xs_reserved_E}{#6}%
	\gdef\groupID{}%
}

\def\exploregroups{\let\xs_exploregroups\relax}
\def\noexploregroups{\def\xs_exploregroups{\let\xs_toks\relax}}
\def\saveexploremode{\let\xs_saveexploremode\xs_exploregroups}
\def\restoreexploremode{\let\xs_exploregroups\xs_saveexploremode}

% Remplace les #1 premières occurences de la chaine #3
% par la chaine #4 dans la chaine #2
\xs_newmacro\StrSubstitute{0}{3}{1}{%
	\def\xs_reserved_D{#2}\let\xs_reserved_C\empty\def\xs_reserved_E{#3}%
	\decimalpart0 \integerpart0 \gdef\groupID{0}\def\xs_nestlevel{0}%
	\def\xs_atoccurfound{%
		\xs_ifnum{#1<1 }
			{\def\xs_reserved_A{#4}}
			{\xs_ifnum{\decimalpart>#1 }
				{\def\xs_reserved_A{#3}}
				{\def\xs_reserved_A{#4}}%
			}%
		\xs_eaddtomacro\xs_reserved_C{\xs_reserved_A}%
	}%
	\xs_ifempty{#3}%
		{\xs_exparg\xs_assign_xstringresult{\xs_reserved_D}{#5}%
		}
		{\xs_read_reserved_D
		\xs_exparg\xs_assign_xstringresult{\xs_reserved_C}{#5}%
		}%
}

% Supprime les #1 premières occurrences de #3 dans #2
\xs_newmacro\StrDel{0}{2}{1}{\xs_StrSubstitute[#1]{#2}{#3}{}[#4]}

\def\xs_exitallgroups{\ifnum\xs_nestlevel>0 \endgroup\expandafter\xs_exitallgroups\fi}

% Compte combien d'unités syntaxiques contient la chaine #1 ?
\xs_newmacro\StrLen{}{1}{1}{%
	\def\xs_reserved_C{#1}\decimalpart0 \def\xs_nestlevel{0}\gdef\groupID{0}%
	\let\xs_atbegingroup\relax
	\def\xs_atendofgroup{\expandafter\endgroup\expandafter\decimalpart\number\decimalpart\relax}%
	\let\xs_atnextsyntaxunit\relax
	\xs_read_reserved_C
	\xs_exparg\xs_assign_xstringresult{\number\decimalpart}{#2}%
}

% Macro interne fermant autant de groupes que nécessaire pour trouver une unité syntaxique derrière
\def\xs_continuetonext{%
	\xs_exparg\xs_ifempty{\xs_reserved_C}%
		{\ifnum\xs_nestlevel>0
			\xs_endgroup_eaddtomacro\xs_reserved_B{\expandafter{\xs_reserved_B}}% on concatène
			\expandafter\xs_continuetonext% on recommence
		\fi}
		{}%
}%

\def\xs_calc_grouIP{%
	\begingroup\def\xs_reserved_A{0}%
	\xs_ifx{\xs_reserved_A\groupID}% si on arrive du groupe de niveau 0
		{\endgroup\xdef\groupID{\number\integerpart}}% on met simplement le niveau
		{\endgroup\xs_eearg{\gdef\groupID}{\expandafter\groupID\expandafter,\number\integerpart}}%
}

% Cette macro coupe la chaine #1 après l'unité syntaxique n°#2
\def\StrSplit{%
	\xs_ifstar
		{\let\xs_reserved_E\xs_continuetonext\StrSplit_i}%
		{\let\xs_reserved_E\relax\StrSplit_i}%
}

\xs_newmacro\StrSplit_i{}{2}{0}{\xs_StrSplit_ii{#2}{#1}\xs_StrSplit_iii}

% Cette macro interne coupe la chaine #2 après l'unité syntaxique n°#1
% Le début est assigné dans \xs_reserved_B et la fin dans \xs_reserved_C
\long\def\xs_StrSplit_ii#1#2{%
	\let\xs_reserved_B\empty\def\xs_reserved_C{#2}\def\xs_reserved_D{#1}\gdef\groupID{}%
	\ifnum#1>0
		\decimalpart0 \integerpart0 \gdef\groupID{0}\def\xs_nestlevel{0}%
		\def\xs_atendofgroup{%
			\edef\__temp{%
				\endgroup
				\noexpand\decimalpart=\number\decimalpart\relax
				\unexpanded{\xs_addtomacro\xs_reserved_B}{{\unexpanded\expandafter{\xs_reserved_B}}}%
				}%
			\__temp
		}%
		\def\xs_atbegingroup{\let\xs_reserved_B\empty}%
		\def\xs_atnextsyntaxunit{%
			\xs_eaddtomacro\xs_reserved_B{\xs_reserved_A}%
			\xs_ifnum{\decimalpart=\xs_reserved_D\relax}
				{\ifx\xs_reserved_C\empty\xs_reserved_E\fi
				\edef\__temp{\def\noexpand\xs_reserved_B{\unexpanded\expandafter{\xs_reserved_B}}\def\noexpand\xs_reserved_C{\unexpanded\expandafter{\xs_reserved_C}}}%
				\expandafter\xs_exitallgroups\__temp
				\xs_gobarg% mange l'appel récursif \xs_read_reserved_C dans \xs_read_reserved_C
				}
				{}%
		}%
		\xs_read_reserved_C
	\fi
}

\def\xs_StrSplit_iii#1#2{\let#1\xs_reserved_B\let#2\xs_reserved_C}

% StrCut[n]{<texte>}{<motif>}\macroA\macroB
% \macroA reçoit ce qui se trouve avant la n ème occurrence <motif> dans <texte>
% \macroB reçoit ce qui est après cette n ème occurrence
\xs_newmacro*3\StrCut{1}{2}{0}{%
	\xs_ifempty{#3}%
		{\gdef\groupID{}\let\xs_reserved_C\empty\let\xs_reserved_E\empty
		}
		{\xs_ifnum{#1<1 }
			{\xs_StrCut_ii{#2}{#3}1}
			{\xs_StrCut_ii{#2}{#3}{#1}}%
		}%
	\xs_StrCut_iii
}

\long\def\xs_StrCut_ii#1#2#3{%
	\let\xs_reserved_C\empty\def\xs_reserved_D{#1}\def\xs_reserved_E{#2}%
	\decimalpart0 \integerpart0 \gdef\groupID{0}\def\xs_nestlevel{0}%
	\def\xs_atoccurfound{%
		\xs_ifnum{\decimalpart=\numexpr(#3)\relax}
			{\edef\__temp{%
				\def\noexpand\xs_reserved_C{\unexpanded\expandafter{\xs_reserved_C}}%
				\def\noexpand\xs_reserved_D{\unexpanded\expandafter{\xs_reserved_D}}%
				\noexpand\decimalpart=\number\decimalpart\relax
				}%
			\expandafter\xs_exitallgroups\__temp
			\xs_gobarg% stoppe la récursivité dans \xs_read_reserved_D
			}
			{\xs_eaddtomacro\xs_reserved_C{\xs_reserved_E}%
			}%
	}%
	\xs_read_reserved_D
	\let\xs_reserved_E\xs_reserved_D
	\xs_exparg{\def\xs_reserved_D}{\xs_reserved_C#2}%
}

\long\def\xs_StrCut_iii#1#2{\let#1\xs_reserved_C\let#2\xs_reserved_E}

% De la chaine #1, renvoie ce qui se trouve entre les positions
%  #2 et #3, unités syntaxiques aux positions compris !
\xs_newmacro\StrMid{}{3}{1}{%
	\begingroup
		\noexploregroups
		\let\xs_reserved_E\relax
		\xs_StrSplit_ii{#3}{#1}%
		\edef\xs_reserved_C{\number\numexpr#2-1}%
		\let\xs_reserved_E\relax
		\xs_exptwoargs\xs_StrSplit_ii\xs_reserved_C\xs_reserved_B
	\expandafter\endgroup
	\expandafter\xs_assign_xstringresult\expandafter{\xs_reserved_C}{#4}%
	\gdef\groupID{}%
}

% supprime #2 unités syntaxiques à gauche dans la chaine #1
\xs_newmacro\StrGobbleLeft{}{2}{1}{%
	\let\xs_reserved_E\relax
	\xs_StrSplit_ii{#2}{#1}%
	\xs_exparg\xs_assign_xstringresult{\xs_reserved_C}{#3}%
}

% extrait de #1 la chaine à gauche de longueur #2
\xs_newmacro\StrLeft{}{2}{1}{%
	\let\xs_reserved_E\relax
	\xs_StrSplit_ii{#2}{#1}%
	\xs_exparg\xs_assign_xstringresult{\xs_reserved_B}{#3}%
}

% supprime #2 unités syntaxiques à droite dans la chaine #1
\xs_newmacro\StrGobbleRight{}{2}{1}{%
	\xs_StrLen{#1}[\xs_reserved_D]%
	\let\xs_reserved_E\relax
	\xs_exparg\xs_StrSplit_ii{\number\numexpr\xs_reserved_D-#2}{#1}%
	\xs_exparg\xs_assign_xstringresult{\xs_reserved_B}{#3}%
}

% renvoie #2 unités syntaxiques à la droite de la chaine #1
\xs_newmacro\StrRight{}{2}{1}{%
	\xs_StrLen{#1}[\xs_reserved_D]%
	\let\xs_reserved_E\relax
	\xs_exparg\xs_StrSplit_ii{\number\numexpr\xs_reserved_D-#2}{#1}%
	\xs_exparg\xs_assign_xstringresult{\xs_reserved_C}{#3}%
}

\xs_newmacro\StrChar{}{2}{1}{%
	\let\xs_reserved_B\empty\def\xs_reserved_C{#1}\def\xs_reserved_D{#2}%
	\ifnum#2>0
		\def\groupID{0}\def\xs_nestlevel{0}\integerpart0 \decimalpart0
		\let\xs_atbegingroup\relax
		\def\xs_atendofgroup{\expandafter\endgroup\expandafter\decimalpart\number\decimalpart\relax}%
		\def\xs_atnextsyntaxunit{%
			\ifnum\decimalpart=\xs_reserved_D% la n ième US est atteinte ?
				\edef\__temp{\def\noexpand\xs_reserved_B{\unexpanded\expandafter{\xs_reserved_A}}}% on capture l'US en cours qui est celle cherchée
				\expandafter\xs_exitallgroups\__temp
				\expandafter\xs_gobarg% stoppe la récursivité dans \xs_read_reserved_C
			\fi
		}%
		\xs_read_reserved_C
	\fi
	\xs_exparg\xs_ifempty{\xs_reserved_B}%
		{\gdef\groupID{}}
		{}%
	\xs_exparg\xs_assign_xstringresult{\xs_reserved_B}{#3}%
}

% Combien de fois compte t-on #2 dans #1 ?
\xs_newmacro\StrCount{}{2}{1}{%
	\def\xs_reserved_D{#1}\def\xs_reserved_E{#2}\let\xs_reserved_C\empty
	\xs_ifempty{#2}%
		{\xs_assign_xstringresult{0}{#3}%
		}
		{\decimalpart0 \integerpart0 \gdef\groupID{0}\def\xs_nestlevel{0}%
		\def\xs_atoccurfound{\let\xs_reserved_C\empty\xs_read_reserved_D}%
		\xs_read_reserved_D
		\xs_exparg\xs_assign_xstringresult{\number\decimalpart}{#3}%
		}%
}

% renvoie la position de l'occurrence #1 de la sous chaine #3 dans la chaine #2
\xs_newmacro\StrPosition{1}{2}{1}{%
	\xs_cutafteroccur{#2}{#3}{#1}%
	\let\xs_reserved_E\groupID
	\xs_ifx{\xs_reserved_C\xs_reserved_D}
		{\xs_assign_xstringresult{0}{#4}%
		\let\xs_reserved_E\empty
		}
		{\xs_exparg\xs_StrLen{\xs_reserved_C}[\xs_reserved_C]%
		\xs_exparg\xs_assign_xstringresult{\number\numexpr\xs_reserved_C+1}{#4}%
		}%
	\global\let\groupID\xs_reserved_E
}

\def\comparestrict{\def\xs_comparecoeff{1}}
\def\comparenormal{\def\xs_comparecoeff{0}}
\def\savecomparemode{\let\xs_saved_comparecoeff\xs_comparecoeff}
\def\restorecomparemode{\let\xs_comparecoeff\xs_saved_comparecoeff}

% Compare les deux arguments #1 et #2
% Renvoie 0 s'ils sont égaux et renvoie la
% position de la premiere unité syntaxiques différente sinon
\xs_newmacro*2\StrCompare{}{2}{1}{%
	\def\xs_reserved_A{#1}%
	\def\xs_reserved_B{#2}%
	\xs_ifx{\xs_reserved_B\xs_reserved_A}
		{\xs_assign_xstringresult{0}{#3}}
		{\xs_StrCompare_i{#1}{#2}{#3}}%
}

\long\def\xs_StrCompare_i#1#2#3{%
	\def\xs_StrCompare_iii##1{%
		\let\xs_reserved_A\empty
		\xs_exparg\xs_ifempty{\xs_reserved_C}%
			{\edef\xs_reserved_A{*\xs_comparecoeff}}%
			{\xs_exparg\xs_ifempty{\xs_reserved_D}%
				{\edef\xs_reserved_A{*\xs_comparecoeff}}
				{}%
			}%
		\def\xs_next{\xs_exparg\xs_assign_xstringresult{\number\numexpr##1\xs_reserved_A\relax}{#3}}%
	}%
	\def\xs_StrCompare_ii##1{% ##1 est la position
		\xs_exparg\xs_returnfirstsyntaxunit{\xs_reserved_C}\xs_reserved_A
		\xs_exparg\xs_returnfirstsyntaxunit{\xs_reserved_D}\xs_reserved_B
		\xs_ifx{\xs_reserved_B\xs_reserved_A}
			{\xs_exparg\xs_ifempty{\xs_reserved_A}%
				{\xs_StrCompare_iii{##1}% les 2 unités syntaxiques sont égales, on renvoie la position
				}
				{\def\xs_next{\xs_exparg\xs_StrCompare_ii{\number\numexpr##1+1}}% les 2 unités syntaxiques sont égales et non vides, on recommence
				\xs_exparg\xs_removefirstsyntaxunit{\xs_reserved_C}\xs_reserved_C
				\xs_exparg\xs_removefirstsyntaxunit{\xs_reserved_D}\xs_reserved_D
				}%
			}% les 2 unités syntaxiques sont différentes : on renvoie la position
			{\xs_StrCompare_iii{##1}%
			}%
		\xs_next
	}%
	\def\xs_reserved_C{#1}\def\xs_reserved_D{#2}%
	\xs_StrCompare_ii1%
}

\xs_newmacro\StrFindGroup{}{2}{1}{%
	\def\xs_reserved_A{#2}\def\xs_reserved_B{0}%
	\xs_ifx{\xs_reserved_A\xs_reserved_B}
		{\xs_assign_xstringresult{{#1}}{#3}}
		{\xs_StrFindGroup_i{#1}{#2}[#3]}%
}

\long\def\xs_StrFindGroup_i#1#2[#3]{%
	\def\xs_StrFindGroup_ii{%
		\xs_exparg\xs_ifempty{\xs_reserved_C}%
			{\xs_assign_xstringresult{}{#3}% s'il ne reste plus rien, on renvoie vide
			}
			{\xs_exparg\xs_returnfirstsyntaxunit{\xs_reserved_C}\xs_reserved_D
			\xs_ifx{\bgroup\xs_toks}% si la 1ère unité syntaxique est un groupe explicite
				{\advance\decimalpart1 % on augmente le compteur
				\xs_ifnum{\decimalpart=\xs_reserved_A\relax}% on est au groupe cherché lors de la profondeur courante ?
					{\xs_ifx{\empty\xs_reserved_B}% on est à la profondeur maximale ?
						{\xs_exparg\xs_assign_xstringresult{\xs_reserved_D}{#3}}% on renvoie ce groupe
						{\xs_exparg{\expandafter\xs_StrFindGroup_i\xs_reserved_D}{\xs_reserved_B}[#3]}% on recommence avec ce groupe et les profondeurs de recherche restantes
					}
					{\xs_exparg\xs_removefirstsyntaxunit{\xs_reserved_C}\xs_reserved_C
					\xs_StrFindGroup_ii
					}%
				}
				{\xs_exparg\xs_removefirstsyntaxunit{\xs_reserved_C}\xs_reserved_C
				\xs_StrFindGroup_ii
				}%
			}%
	}%
	\xs_extractgroupnumber{#2}\xs_reserved_A\xs_reserved_B
	\decimalpart0
	\xs_ifnum{\xs_reserved_A>0 }
		{\def\xs_reserved_C{#1}}
		{\let\xs_reserved_C\empty}%
	\xs_StrFindGroup_ii
}

\def\xs_extractgroupnumber#1#2#3{%
	\def\xs_extractgroupnumber_i##1,##2\_nil{\def#2{##1}\def#3{##2}}%
	\xs_extractgroupnumber_i#1,\_nil
	\ifx\empty#3\else\xs_extractgroupnumber_i#1\_nil\fi
}

\def\expandingroups{\let\xs_expandingroups\exploregroups}

\def\noexpandingroups{\let\xs_expandingroups\noexploregroups}

\def\StrExpand{\xs_testopt{\xs_StrExpand}{1}}
\long\def\xs_StrExpand[#1]#2#3{%
	\begingroup
		\xs_expandingroups
		\xs_ifnum{#1>0 }
			{\integerpart#1 \decimalpart0 \gdef\groupID{0}\def\xs_nestlevel{0}%
			\def\xs_atendofgroup{\xs_endgroup_eaddtomacro\xs_reserved_B{\expandafter{\xs_reserved_B}}}%
			\def\xs_atbegingroup{\let\xs_reserved_B\empty}%
			\def\xs_atnextsyntaxunit{\xs_eeaddtomacro\xs_reserved_B{\xs_reserved_A}}%
			\def\xs_reserved_C{#2}%
			\xs_StrExpand_i{#1}% Appel de la macro récursive
			}
			{\def\xs_reserved_B{#2}%
			}
		\expandafter
	\endgroup
	\expandafter\def\expandafter#3\expandafter{\xs_reserved_B}%
	\gdef\groupID{}%
}

\long\def\xs_StrExpand_i#1{%
	\xs_ifnum{#1>0 }
		{\let\xs_reserved_B\empty
		\xs_read_reserved_C
		\let\xs_reserved_C\xs_reserved_B
		\def\xs_reserved_A{\xs_exparg\xs_StrExpand_i{\number\numexpr#1-1}}%
		}
		{\let\xs_reserved_A\relax
		}%
	\xs_reserved_A
}

\def\scancs{\xs_testopt{\xs_scancs}{1}}
\long\def\xs_scancs[#1]#2#3{%
	\xs_StrExpand[#1]{#3}{#2}%
	\edef#2{\detokenize\expandafter{#2}}%
}

\restorexstringcatcode

\setverbdelim{|}%
\fullexpandarg\saveexpandmode
\comparenormal\savecomparemode
\noexploregroups\saveexploremode
\expandingroups

\endinput

                        H I S T O R I Q U E
------------------------------------------------------------------------------
v1.0    4/7/2008 : - Package achevé.
         7/7/2008 : - Package publié sur CTAN.
------------------------------------------------------------------------------
v1.1    15/8/2008
   - Le package ne nécessite plus LaTeX et est désormais utilisable sous
     Plain eTeX.
   - Ajout des macros \IfInteger et \IfDecimal.
     Pour cette dernière, la partie entière et la partie décimale sont
     récupérables dans les compteurs \integerpart et \decimalpart. Ce
     qu'il reste après qu'un caractère non attendu soit rencontré est
     dispo dans \xs_remainafterdecimal
   - Correction d'un bug dans \tokenize :
        suppression de l'expansion avec \normalexpandarg pour éviter le bug
        avec les lettres accentuées dans l'exemple : \tokenize\aa{éù}
   - Amélioration du code de la plupart des macros.
   - écriture d'une routine \xs_newmacro définissant les  macros.
   - Suppression de tous les compteurs grace à \numexpr.
   - Nouvelles fonctionnalités internes dans :
      \IfSubStr
        ajout des chaines se trouvant avant et après l'occurrence testée
        dans les sc \xs_behind_IfSubStr et \xs_before_IfSubStr
      \StrCount
        a) ajout de ce qui se trouve après la dernière occurrence dans la
           sc \xs_StrCount_afterlastocccur
        b) la macro ne dépend plus d'aucune autre
------------------------------------------------------------------------------
v1.2    23/8/2008
   - Amélioration du code de \StrMid \StrGobbleLeft, \StrLeft,
     \StrGobbleRight et \StrRight à l'aide de la nouvelle macro
     \xs_StrSplit_i
   - Nouveau test \IfStrEq qui compare 2 chaines
   - Nouveau test \IfEq qui compare 2 chaines (ou 2 nombres si les 2 chaines
     contiennent un nombre)
   - Correction d'un bug dans \xs_formatnumber. La macro n'envisageait pas le
     cas où son argument était vide ce qui faiasait planter
     \IfDecimal{}{V}{F} et \IfInteger{}{V}{F}
------------------------------------------------------------------------------
v1.3    21/9/2008
  - Amélioration du code, mise en place de récursivités terminales dans
    toutes les macros récursives, simplification de \xs_TestEqual en
    utilisant \ifx
  - Nouvelles macros \IfStrEqCase et \IfEqCase
      Teste si une chaîne est égale à plusieurs chaines données et
      exécute un code spécifique
  - Création de la macro publique \StrSplit
      Coupe la chaine en 2 à une position donnée
  - Nouvelle macro \StrCompare
      Compare 2 chaînes et renvoie la position du 1er caractère différent.
      Renvoie 0 si elles sont égales.
------------------------------------------------------------------------------
v1.3a   29/9/2008
  - Correction d'un bug : \expandafter#1
    dans les macros \IfSubStrBefore et \StrBetween qui buggue si #1 est un
    nombre > 9 !
  - Ré écriture de \xs_returnfirstsyntaxunit et \xs_gobarg qui
    deviennent purement développables.
  - Ré écriture de \StrLen
  - Nouvelle macro interne \xs_cutafteroccur
      Coupe la chaine après la nième occurrence d'une sous chaine.
  - Code amélioré de \StrMid, \StrLeft, \StrGobbleLeft, \StrRight,
    \StrGobbleRight grace à \xs_cutafteroccur
------------------------------------------------------------------------------
v1.4   4/11/2008
  - xstring traite ses arguments par unité syntaxique, et donc toutes les
    macros internes ont été ré-écrites à cette fin.
  - Le manuel a été ré-écrit, rendu plus clair et mieux structuré.
  - Nouvelle fonctionnalité : \expandarg
      le 1er lexème de chaque argument est 1-développé avant que la
      macro ne soit appelée. Si un argument comporte plus qu'1 lexème,
      il faut passer par \scancs*[1] pour 1-développer toutes les unités
      syntaxiques.
  - Correction de 2 erreurs dans le manuel anglais :  0 est la valeur par
    défaut dans \StrSubstitute et \StrDel
  - Cohérence dans les nom des séquences de contrôles temporaires qui
    sont désormais : \xs_reserved_A à \xs_reserved_E
------------------------------------------------------------------------------
v1.4a  12/11/2008
  - Correction d'un bug dans \xs_StrSplit dans le cas où l'argument
     numérique est <1 : la 1ere coupure doit être vide et le seconde doit
     être égale à l'argument.
------------------------------------------------------------------------------
v1.4b  16/11/2008
  - Correction de 2 bugs dans \xs_returnfirstsyntaxunit
------------------------------------------------------------------------------
v1.5   31/12/2008
  - Nouvelles fonctionnalités dans la recherche de groupes par un
    identificateur caractérisant leur imbrication. Implémentation de :
      1) \StrFindGroup
      2) \groupID
  - Ajout des nouvelles macros génériques \xs_read_reserved_C et
    \xs_read_reserved_D
  - Correction d'un bug dans \StrSplit : \StrSplit{{a}bc}{1}[\truc] faisait
    que \truc contenait «a{a}».
  - Correction d'un bug dans \xs_newmacro : l'argument optionnel était
    détokénisé à tort dans les macros étoilées.
  - Dans \xs_newmacro, on doit choisir le numéro du dernier argument devant
    être détokénisé dans le cas d'une macro étoilée (sachant que
     l'argument optionnel ne l'est jamais)
------------------------------------------------------------------------------
v1.5a  09/2/2009
  - Nouvelle macro \StrExpand, sensible au mode d'exploration. La macro est
    indépendante du mode d'exploration et dispose de ses propres commandes
    d'exploration : \expandingroups et \noexpandingroups
  - Ré-écriture de \scancs, qui devient triviale puisque dépendant de
    \StrExpand. Il n'y a plus de version étoilée et désormais, \scancs
    est sensible au mode d'exploration.
  - Correction d'un bug dans \IfInteger : \IfInteger{2.0}{V}{F} affichait F
  - Correction de bugs : mettre \let\xs_reserved_E\relax avant d'appeler
    \xs_StrSplit_i
  - Suppression des messages d'erreur, de warning et d'info. xstring devient
    donc silencieux ;)
  - Passage de 3 à 2 compteurs
  - Le manuel est désormais plus coloré, en espérant qu'il sera un peu
    plus agréable à lire !
------------------------------------------------------------------------------
v1.5b  13/3/2009
  - Modification d'un grand nombre de macros de telle sorte que vraiment
    n'importe quel code peut désormais être placé dans les arguments des
    macros (comme par exemple des \if, \ifx ou autre, même sans leur \fi, ce
    qui n'était pas possible jusqu'alors)
------------------------------------------------------------------------------
v1.5c  05/6/2009
  - Correction d'un bug dans \IfBeginWith : il faut appeler
    \xs_returnfirstsyntaxunit pour l'argument #2 puis pour l'argument #1 et
    non pas l'inverse pour que \xs_toks reflète ce qui se trouve dans #1
  - Correction d'un bug dans \xs_returnfirstsyntaxunit au cas où #1 commence
    par un espace : mettre \xs_toks à 0
  - Correction d'un bug dans \xs_returnfirstsyntaxunit :
    \xs_returnfirstsyntaxunit{{}}\truc
    faisait que \truc se développait en un espace.
  - Correction d'un bug dans \xs_removefirstsyntaxunit où le cas
    \xs_removefirstsyntaxunit{ {b}}\truc
    donnait \truc se développant en «b» au lieu de «{b}».
------------------------------------------------------------------------------
v1.5d  28/3/2010
  - Correction d'un bug dans \IfInteger : \afterinteger ne donnait pas le
    bon résultat. Par exemple avec 1.23 il renvoyait {} au lieu de {.23}
  - \xs_formatnumber ne detokenize plus ses arguments (que m'est-il passé
    par la tête pour que je code un truc comme ca ?), et donc
    \afterinteger et \xs_afterdecimal ont des tokens ayant le bon
    catcode
------------------------------------------------------------------------------
v1.6  24/10/2012
  - Correction d'un bug dans \xs_cutafteroccur au cas où #3<1
  - \IfDecimal ne \detokenize plus l'argument.
  - Nouvelle version de \tokenize qui enlevait les accolades si son argument
    était entre accolades
  - Nouvelle macro \StrCut
  - \xs_cutafteroccur_i ne produit plus d'espace indésirable si #3 était
    évalué par \numexpr
------------------------------------------------------------------------------
v1.7  13/1/2013
  - Correction d'un manque de rigueur dans les assignations concernant
    \groupID : elles sont toutes globales désormais
  - Les macros de xstring sont \long
------------------------------------------------------------------------------
v1.7a 28/2/2013
  - Correction d'un espace parasite dans \xs_removefirstsyntaxunit
------------------------------------------------------------------------------
v1.7b 29/7/2013
  - La date contenue dans \xstringdate est désormais conforme
------------------------------------------------------------------------------
v1.7c 13/10/2013
  - les macros impliquées dans \IfStrCase et \IfStrEqCase sont désormais \long
------------------------------------------------------------------------------
v1.8  07/12/2018
  - affectations globales supprimées (sauf pour \groupID) afin de ne pas
    surcharger la pile de sauvegarde
  - nettoyage du code, corrections de lourdeurs et maladresses
------------------------------------------------------------------------------
v1.81 09/12/2018
  - bug dans \IfInteger et \IfDecimal, la macro \xs_afterdecimal est indéfinie
    et doit être remplacée par \afterdecimal
------------------------------------------------------------------------------
v1.82 01/02/2019
  - \xs_testopt n'est pas longue et provoque un bug dans \xs_IfStrCase et
    \xs_IfStrCase lorsque #1 contient \par : \xs_testopt reste courte et
    \xs_IfStringCase a été modifiée pour que cela fonctionne.
    ------------------------------------------------------------------------------
v1.83 06/02/2019
  - mauvaise approche et mauvais fix dans la v1.82 : \xs_testopt devient
    longue, pas le choix (et donc aussi \xs_ifnxttok)
------------------------------------------------------------------------------
v1.84 21/07/2021
  - encodage du fichier xstring.tex en UTF-8
